% \documentclass{article}
% \usepackage[todonotes]{newrevisor}
% \usepackage{marginnote}
% \let\marginpar\marginnote
% \setlength{\marginparwidth}{3cm}
\usepackage[a5paper,paperheight=6cm,left=0cm,right=3.5cm,top=0cm,bottom=0cm,nohead]{geometry}
% First argument is the desired command name, second argument is color for
% additions, third optional argument is color for deletions. If the third
% argument is missing, the second argument is used for both additions and
% deletions.  See colornames in https://en.wikibooks.org/wiki/LaTeX/Colors
% This example creates lowercase \manuel{TEXT-TO-DELETE}{TEXT-TO-ADD} and uppercase \MANUEL{COMMENT TEXT}. Use the starred version \MANUEL*{COMMENT TEXT} to force the comment to be inline (for footnotes, captions, tables, etc.)
\newrevisor{manuel}{violet!75}
\newrevisor{maria}{blue}
\newrevisor{peter}{green!80!black}[red]
% \hiderevisor{manuel}
% \hiderevisor{maria}
\hypersetup{colorlinks}

\begin{document}
\thispagestyle{empty}
This is \manuel{text-to-be-deleted}{text-to-be-added}.\MANUEL{This is a note}
\footnote{This is a footnote.\MANUEL*{this is a note in the footnote}}

This is \maria{text-to-be-deleted}{text-to-be-added}.\MARIA{This is a note}
\footnote{This is a footnote.\MARIA*{this is a note in the footnote}}

This is \peter{text-to-be-deleted}{text-to-be-added}.\PETER{This is a note}
\footnote{This is a footnote.\PETER*{this is a note in the footnote}}

\end{document}
%%% Local Variables:
%%% mode: latex
%%% TeX-master: t
%%% End:
