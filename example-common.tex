\usepackage[a5paper,paperheight=20cm,left=0cm,right=3.5cm,top=0cm,bottom=0cm,nohead]{geometry}
\usepackage[colorlinks]{hyperref}
\newrevisor{manuel}{violet!75}
\newrevisor{maria}{blue}
\newrevisor{peter}{green!80!black}[red]
\begin{document}
\thispagestyle{empty}
\listofchanges%

This is \manuel{text-to-be-deleted}{text-to-be-added}.\MANUEL{This is a note}
\footnote{This is a footnote.\MANUEL*{this is a note in the footnote}}

This is \maria{text-to-be-deleted}{text-to-be-added}.\MARIA{This is a note}
\footnote{This is a footnote.\MARIA*{this is a note in the footnote}}

This is \peter{text-to-be-deleted}{text-to-be-added}.\PETER{This is a note}
\footnote{This is a footnote.\PETER*{this is a note in the footnote}}

\peter{Old line 1\par Old line 2}{New line 1\par New line 2}

\begin{figure}[h!]
  \centering
  A FIGURE
  \caption{Caption \manuel{and}{with} \PETER{not inline}\MANUEL*{an inline note}}
\end{figure}

\section[test]{A note in a \peter{chapter}{section} \MANUEL*{inline sorry!} \PETER{fixed}}

Some math:
\begin{equation}
  E = \maria{n}{m}c^{\peter{3}{2}} \MANUEL*{Is this correct?}\MARIA{Sure!}
\end{equation}

\MARIA*{A long note with some bullet points:
  \begin{itemize}
  \item Do 1
  \item Do 2
    \item Profit!
    \end{itemize}
  }
\end{document}

%%% Local Variables:
%%% mode: latex
%%% TeX-master: t
%%% End:
